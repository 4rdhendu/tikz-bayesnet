%% example.tex
%% Copyright (C) 2010,2011 Laura Dietz
%% Copyright (C) 2012 Jaakko Luttinen
%
% This work may be distributed and/or modified under the conditions of
% the LaTeX Project Public License, either version 1.3 of this license
% or (at your option) any later version.  The license is in the file
% LICENSE and the latest version of this license is in
% http://www.latex-project.org/lppl.txt and version 1.3 or later is
% part of all distributions of LaTeX version 2005/12/01 or later.
%
% This work has the LPPL maintenance status `maintained'.
% 
% The Current Maintainer of this work is Jaakko Luttinen.
%
% This work consists of the files bayesnet.sty and example.tex.

\documentclass[a4paper]{article}

\usepackage{bayesnet}

\title{Graphical Models in Tikz}
\author{Laura Dietz, Jaakko Luttinen}

\begin{document}

\maketitle

TikZ examples for graphical models (Bayesian networks) and factor
graphs.


\begin{figure}[ht]
  \begin{center}
    \begin{tikzpicture}
      % Define nodes
      \node[const] (a) at (0,3) {$\alpha$};
      \node[latent] (w) at (0,2) {$\mathbf{w}_n$};
      \node[latent] (x) at (2,2) {$\mathbf{x}_n$};
      \node[latent] (t) at (4,0) {$\tau$};
      \node[obs] (y) at (1,0) {$\mathbf{y}_n$};

      % Connect the nodes
      \draw[->] (a) -- (w);
      \draw[->] (x) -- (y);
      \draw[->] (w) -- (y);
      \draw[->] (t) -- (y);

      % Plates
      \plate{N}{(x)(y)(w)}{$n=1,\ldots,N$}{}
    \end{tikzpicture}
  \end{center}
  \caption{A simple Bayesian network.}
\end{figure}


% Latent Dirichlet allocation
\begin{figure}[ht]
  \begin{center}
    \begin{tikzpicture}

      % Nodes
      
      \matrix[row sep=15pt,column sep=40pt, matrix anchor=mid] (lda) {
        &
        \\
        &
        \\
        \node (lambda) [latent] {$\theta$}; &
        \\
        \nofactor{drawtc}{}{Multi}{left=0pt}; &
        \\
        \node (tc) [latent] {$T$}; &
        \\
        \nofactor{drawxc}{}{Multi}{above=0pt}; \gate{gatexc}{(drawxc)
          (captdrawxc)}{$t$}; & \node (phi) [latent] {$\phi$};
        \\
        \node (xc) [obs] {$X$}; &
        \\
      };
      \nofactor{drawlambda}{above=15pt of lambda}{Dir}{right=0pt};
      \nofactor{drawphi}{above=15pt of phi}{Dir}{right=0pt};
      \node (alphalambda) [const, above=15pt of drawlambda]  {$\alpha_{\theta}$};
      \node (alphaphi) [const, above=15pt of drawphi]  {$\alpha_{\phi}$};

      % Connections
      
      \draw [->] (alphaphi) -- (drawphi) -- (phi);
      \draw [->] (alphalambda) -- (drawlambda) -- (lambda);
      \draw [->] (lambda) -- (drawtc) -- (tc);
      \draw [->] (phi) -- (drawxc) -- (xc);
      \draw [-*] (tc) -- (gatexc);

      % Plates

      \plate {contentplatec}
      {(tc)(xc)(drawtc)(drawxc)(captdrawtc)(captdrawxc)(gatexc)}
      {$\forall 1 \leq i \leq n_d $} {}

      \plate {userplatec}
      {(contentplatec)(lambda)(drawlambda)(captdrawlambda)} {$\forall
        d \in \mathcal{D}$} {}

      \plate {topicplate} {(phi)(drawphi)(captdrawphi)} {$\forall t
        \in \mathcal{T}$}{}

    \end{tikzpicture}

  \end{center}

  \caption{Latent Dirichlet allocation as directed factor graph.}

\end{figure}


\end{document}
