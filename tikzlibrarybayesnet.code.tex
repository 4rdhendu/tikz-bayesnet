% Copyright 2012 by Jaakko Luttinen
%
% This file may be distributed and/or modified
%
% 1. under the LaTeX Project Public License and/or
% 2. under the GNU Public License.
%
% See the file doc/generic/pgf/licenses/LICENSE for more details.

% \ProvidesFileRCS[v\pgfversion] $Header: /cvsroot/pgf/pgf/generic/pgf/frontendlayer/tikz/libraries/tikzlibrarybayesnet.code.tex,v 0.1 2012/09/14 10:30:36 tantau Exp $

% Load other libraries
\usetikzlibrary{shapes,decorations,shadows}
\usetikzlibrary{decorations.pathmorphing}
\usetikzlibrary{decorations.shapes}
\usetikzlibrary{fadings}
\usetikzlibrary{patterns}
\usetikzlibrary{calc}
\usetikzlibrary{decorations.text}
\usetikzlibrary{decorations.footprints}
\usetikzlibrary{decorations.fractals}
\usetikzlibrary{shapes.gates.logic.IEC}
\usetikzlibrary{shapes.gates.logic.US}
\usetikzlibrary{fit,chains}
\usetikzlibrary{positioning}
\usepgflibrary{shapes}
\usetikzlibrary{scopes}
\usetikzlibrary{arrows}
\usetikzlibrary{backgrounds}

% Latent node
\tikzstyle{latent} = [circle,fill=white,draw=black,inner sep=1pt,
minimum size=20pt, font=\fontsize{10}{10}\selectfont, node distance=1]
% Observed node
\tikzstyle{obs} = [latent,fill=gray!25]
% Constant node
\tikzstyle{const} = [rectangle, inner sep=0pt, node distance=1]
% Factor node
\tikzstyle{factor} = [rectangle, fill=black,minimum size=5pt, inner
sep=0pt, node distance=0.4, font=\small]

%\tikzstyle{label} += [font=\footnotesize]

% Caption node?
%\tikzstyle{caption} = [node distance=2pt]

\tikzset{plates/.code={\text{Hello world!}}}

% Plate node
\tikzstyle{plate} = [draw, rectangle, rounded corners, fit=#1]
% Constant node
\tikzstyle{wrap} = [inner sep=0pt, fit=#1]
% Caption node
\tikzstyle{caption} = [font=\footnotesize] %
\tikzstyle{plate caption} = [caption, node distance=0, inner sep=0pt,
below left=5pt and 0pt of #1.south east] %
\tikzstyle{factor caption} = [caption] %
\tikzstyle{every label} += [caption] %

\tikzset{>={triangle 45}}

\pgfdeclarelayer{b}
\pgfdeclarelayer{f}
\pgfsetlayers{b,main,f}

% name, fitlist, caption
\newcommand{\newplate}[3][default]{ %
  \node[wrap=#2] (#1-wrap) {}; %
  \node[plate caption=#1-wrap] (#1-caption) {#3}; %
  \node[plate=(#1-wrap)(#1-caption)] (#1) {}; %
}

% name,  fitlist, input
\newcommand{\newgate}[3][default]{ %
  \node[rectangle,draw,dashed, inner sep=2pt, fit=#2] (#1) {}; %
  \draw [-*,thick] (#3) -- (#1); %
}

%\newcommand{\newfactor}[3][default]{ %
%  \node[factor]
%  }


% shapename, fitlist, caption, pos
\newcommand{\plate}[4]{ %
  \begin{pgfonlayer}{b}
    \node (invis#1) [draw, color=white, inner sep=1pt, rectangle,
    fit=#2] {};
  \end{pgfonlayer}
  \begin{pgfonlayer}{f}
    \node (capt#1) [ below left=0 pt of invis#1.south east,
    xshift=2pt,yshift=1pt] {\footnotesize{#3}}; %
    \node (#1) [draw,inner sep=2pt, rectangle,fit=(invis#1)
    (capt#1),#4] {}; %
  \end{pgfonlayer}
}


\newcommand{\shiftedplate}[5]{ %
  \begin{pgfonlayer}{b}
    \node (invis#1) [draw, color=white, inner sep=0pt, rectangle,
    fit=#2] {};
  \end{pgfonlayer}
  \begin{pgfonlayer}{f}
    \node (capt#1) [#5, xshift=2pt] {\footnotesize{#3}}; %
    \node (#1) [draw,inner sep=2pt, rectangle, fit=(invis#1) (capt#1),
    #4] {}; %
\end{pgfonlayer}
}

%shapename, pos, caption, in1, in2, out, captpos
\newcommand{\twofactor}[7]{ %
  \node (#1) [factor] at #2 {};%
  \node (capt#1) [#7 of #1]{\footnotesize{#3}}; %
  \draw [-] (#4) -- (#1) ; %
  \draw [-] (#5) -- (#1) ; %
  \draw [->,thick] (#1) -- (#6); %
}

%shapename, pos, caption, in, out, captpos
\newcommand{\factorexperimental}[6]{ %
  \node [factor] (#1) at #2 {}; %
  \node (#1-caption) [#6] at #2 {\footnotesize{#3}}; %
  \draw [-] (#4) -- (#1) ; %
  \draw [->,thick] (#1) -- (#5); %
}

% shapename, pos, caption, in, out, captpos
\newcommand{\factor}[6]{ %
  \node (#1) [factor] at #2 {}; %
  \node (#1-caption) [#6 of #1]{\footnotesize{#3}}; %
  \draw [-] (#4) -- (#1) ; %
  \draw [->,thick] (#1) -- (#5); %
}

% name, --, caption, pos
\newcommand{\nofactor}[4]{ %
  \node (#1) [factor, #2] {}; %
  \node (capt#1) [#4 of #1]{\footnotesize{#3}}; %
}

%shapename,  fitlist, caption
\newcommand{\namedgate}[3]{ %
  \begin{pgfonlayer}{b}
    \node (invisgate#1) [rectangle, draw, color=white, fit=#2] {};
  \end{pgfonlayer}
  \node (gatecapt#1) [ above right=0 pt of invisgate#1.north west,
  xshift=-1pt ] {\footnotesize{#3}}; %
  \node (#1) [rectangle,draw,dashed, inner sep=2pt,
  fit=(invisgate#1)(gatecapt#1)]{}; %
}

%shapename,  fitlist, caption
\newcommand{\gate}[3]{ %
  \node (#1) [rectangle,draw,dashed, inner sep=2pt, fit=#2]{};
}

%shapename,  fitlist1, fitlist2, caption1, caption2
\newcommand{\vertgate}[5]{ %
  \begin{pgfonlayer}{b}
    \node (invisgateleft#1) [rectangle, draw, color=white, fit=#2]
    {}; %
    \node (invisgateright#1) [rectangle, draw, color=white, fit=#4]
    {}; %
  \end{pgfonlayer}
  \node (gatecaptleft#1) [ above left=0 pt of invisgateleft#1.north
  east, xshift=1pt ]{\footnotesize{#3}} ; %
  \node (gatecaptright#1) [ above right=0 pt of invisgateright#1.north
  west, xshift=-1pt ] {\footnotesize{#5}}; %
  \node (#1) [rectangle,draw,dashed, inner sep=2pt,
  fit=(invisgateleft#1)(gatecaptleft#1)(invisgateright#1)(gatecaptright#1)]{}; %
  \draw [-, dashed] (#1.north) -- (#1.south); %
}


\newcommand{\vertgateSpec}[5]{
  \begin{pgfonlayer}{b}
    \node (invisgateleft#1) [rectangle, draw, color=white,  fit=#2] {}; %
    \node (invisgateright#1) [rectangle, draw, color=white,  fit=#4] {}; %
  \end{pgfonlayer}
  \node (gatecaptleft#1) [ above left=0 pt of invisgateleft#1.north
  east, xshift=1pt ]{\footnotesize{#3}}; %
  \node (gatecaptright#1) [ above right=0 pt of invisgateright#1.north
  west, xshift=-1pt ] {\footnotesize{#5}}; %
  \node (#1) [rectangle,draw,dashed, inner sep=2pt,
  fit=(invisgateleft#1)(gatecaptleft#1)(invisgateright#1)(gatecaptright#1)]{}; %
  \draw [-, dashed] (#1.70) -- (#1.290); %
}

\newcommand{\horgate}[5]{
  \begin{pgfonlayer}{b}
    \node (invisgateleft#1) [rectangle, draw, color=white, fit=#2]
    {}; %
    \node (invisgateright#1) [rectangle, draw, color=white, fit=#4]
    {}; %
  \end{pgfonlayer}
  \node (gatecaptleft#1) [ above right=0 pt of invisgateleft#1.south
  west, xshift=1pt ]{\footnotesize{#3}}; %
  \node (gatecaptright#1) [ below right=0 pt of invisgateright#1.north
  west, xshift=-1pt ] {\footnotesize{#5}}; %
  \node (#1) [rectangle,draw,dashed, inner sep=2pt,
  fit=(invisgateleft#1)(gatecaptleft#1)(invisgateright#1)(gatecaptright#1)]{}; %
  \draw [-, dashed] (#1.west) -- (#1.east); %
}

\newcommand{\horogate}[5]{
  \begin{pgfonlayer}{b}
    \node (invisgateleft#1) [rectangle, draw, color=white, fit=#2]
    {}; %
    \node (invisgateright#1) [rectangle, draw, color=white, fit=#4]
    {}; %
  \end{pgfonlayer}
  \node (#1) [rectangle,draw,dashed, inner sep=2pt,
  fit=(invisgateleft#1)(invisgateright#1)]{}; %
  \node (gatecaptleft#1) [ above right=0 pt of #1.west, xshift=0pt
  ]{\footnotesize{#3}}; %
  \node (gatecaptright#1) [ below right=0 pt of #1.west, xshift=0pt ]
  {\footnotesize{#5}}; %
  \draw [-, dashed] (#1.west) -- (#1.east); %
}


\newcommand{\vertogate}[5]{
  \begin{pgfonlayer}{b}
    \node (invisgateleft#1) [rectangle, draw, color=white, fit=#2]
    {}; %
    \node (invisgateright#1) [rectangle, draw, color=white, fit=#4]
    {}; %
  \end{pgfonlayer}
  \node (#1) [rectangle,draw,dashed, inner sep=2pt,
  fit=(invisgateleft#1)(invisgateright#1)]{}; %
  \node (gatecaptleft#1) [ below left=0 pt of #1.north, xshift=0pt
  ]{\footnotesize{#3}}; %
  \node (gatecaptright#1) [ below right=0 pt of #1.north, xshift=0pt ]
  {\footnotesize{#5}}; %
  \draw [-, dashed] (#1.north) -- (#1.south); %
}

\endinput

% \tikzset{ %
%   latent/.style={circle,fill=white,draw=black,inner sep=1pt, minimum
%     size=20pt, font=\fontsize{10}{10}\selectfont}, %
%   obs/.style={latent,fill=gray!25}, %
%   const/.style={rectangle, inner sep=0pt}, %
%   factor/.style={rectangle, fill=black,minimum size=5pt, inner
%     sep=0pt}, %
%   >={triangle 45}%
% }



% Styles for states:

\usetikzlibrary{shapes.multipart}

\tikzstyle{every state}=           []

\tikzstyle{state without output}=  [circle,draw,minimum size=2.5em,every state]
\tikzstyle{state with output}=     [circle split,draw,minimum size=2.5em,every state]

\tikzstyle{accepting by arrow}=    [after node path=
{
  {
    [to path=
    {
      [->,double=none,every accepting by arrow]
      --
      ([shift=(\tikz@accepting@angle:\tikz@accepting@distance)]\tikztostart.\tikz@accepting@angle)
          node [shape=rectangle,anchor=\tikz@accepting@anchor] {\tikz@accepting@text}
      }]
    edge ()
  }
}]
\tikzstyle{every accepting by arrow}=[]
\tikzstyle{accepting by double}= [double,outer sep=.5\pgflinewidth+.3pt] % .3pt is half double width distance

\tikzstyle{initial by arrow}=   [after node path=
{
  {
    [to path=
    {
      [->,double=none,every initial by arrow]
      ([shift=(\tikz@initial@angle:\tikz@initial@distance)]\tikztostart.\tikz@initial@angle)
          node [shape=rectangle,anchor=\tikz@initial@anchor] {\tikz@initial@text}
        -- (\tikztostart)}]
    edge ()
  }
}]
\tikzstyle{every initial by arrow}=[]

\tikzstyle{initial by diamond}=[shape=diamond]


\tikzoption{initial text}{\tikzaddafternodepathoption{\def\tikz@initial@text{#1}}}
\tikzoption{accepting text}{\tikzaddafternodepathoption{\def\tikz@accepting@text{#1}}}

\tikzoption{initial where}{\tikzaddafternodepathoption{\csname tikz@initial@compute@#1\endcsname}}
\tikzoption{accepting where}{\tikzaddafternodepathoption{\csname tikz@accepting@compute@#1\endcsname}}

\tikzoption{initial distance}{\tikzaddafternodepathoption{\def\tikz@initial@distance{#1}}}
\tikzoption{accepting distance}{\tikzaddafternodepathoption{\def\tikz@accepting@distance{#1}}}

\def\tikz@initial@text{start}
\def\tikz@accepting@text{}

\def\tikz@initial@distance{3ex}
\def\tikz@accepting@distance{3ex}

\def\tikz@initial@compute@above{\def\tikz@initial@angle{90}\def\tikz@initial@anchor{south}}
\def\tikz@initial@compute@below{\def\tikz@initial@angle{270}\def\tikz@initial@anchor{north}}
\def\tikz@initial@compute@left{\def\tikz@initial@angle{180}\def\tikz@initial@anchor{east}}
\def\tikz@initial@compute@right{\def\tikz@initial@angle{0}\def\tikz@initial@anchor{west}}

\def\tikz@initial@angle{180}
\def\tikz@initial@anchor{east}

\def\tikz@accepting@compute@above{\def\tikz@accepting@angle{90}\def\tikz@accepting@anchor{south}}
\def\tikz@accepting@compute@below{\def\tikz@accepting@angle{270}\def\tikz@accepting@anchor{north}}
\def\tikz@accepting@compute@left{\def\tikz@accepting@angle{180}\def\tikz@accepting@anchor{east}}
\def\tikz@accepting@compute@right{\def\tikz@accepting@angle{0}\def\tikz@accepting@anchor{west}}

\def\tikz@accepting@angle{0}
\def\tikz@accepting@anchor{west}


\tikzstyle{initial above}=        [initial by arrow,initial where=above]
\tikzstyle{initial below}=        [initial by arrow,initial where=below]
\tikzstyle{initial left}=         [initial by arrow,initial where=left]
\tikzstyle{initial right}=        [initial by arrow,initial where=right]

\tikzstyle{accepting above}=      [accepting by arrow,accepting where=above]
\tikzstyle{accepting below}=      [accepting by arrow,accepting where=below]
\tikzstyle{accepting left}=       [accepting by arrow,accepting where=left]
\tikzstyle{accepting right}=      [accepting by arrow,accepting where=right]


% Defaults:

\tikzstyle{state}=                [state without output]
\tikzstyle{accepting}=            [accepting by double]
\tikzstyle{initial}=              [initial by arrow]



\endinput